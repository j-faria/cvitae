\documentclass[10pt]{moderncv}

% moderncv styles
% optional argument are 'blue' (default), 'orange', 'red', 'green', 'grey', 'nocolor' (for B&W) and 'roman' (for roman fonts, instead of sans serif fonts)
%\moderncvstyle{casual} 
\moderncvtheme[grey]{classic}

\usepackage[portuguese]{babel}	% replace by the encoding you are using
\usepackage[utf8]{inputenc}

\usepackage{marvosym}		% Package for the @ "at" symbol

% personal data (the given example is exhaustive; just give what you want)
\firstname{João}
\familyname{Faria}
\title{Curriculum Vit\ae{}}


%\photo[64pt]{foto_cv.jpg}  

%\renewcommand{\listsymbol}{{\fontencoding{U}\fontfamily{ding}\selectfont\tiny\symbol{'102}}} % define another symbol to be used in front of the list items

% the ConTeXt symbol
\def\ConTeXt{%
  C%
  \kern-.0333emo%
  \kern-.0333emn%
  \kern-.0667em\TeX%
  \kern-.0333emt}

% slanted small caps (only with roman family; the sans serif font doesn't exists :-()
%\usepackage{slantsc}
%\DeclareFontFamily{T1}{myfont}{}
%\DeclareFontShape{T1}{myfont}{m}{scsl}{ <-> cork-lmssqbo8}{}
%\usefont{T1}{myfont}{m}{scsl}Testing the font


% command and color used in this document, independently from moderncv 
\definecolor{sectiontitlecolor}{rgb}{1,0,0}% for web links

\newcommand{\up}[1]{\ensuremath{^\textrm{\scriptsize#1}}}% for text subscripts

% teste para fazer lista dentro do cventry
\newcommand\mybitem[1]{%
   \parbox[t]{3mm}{\textbullet}\parbox[t]{10cm}{#1}\\[1.6mm]}

%----------------------------------------------------------------------------------
%            content
%----------------------------------------------------------------------------------
\begin{document}

\maketitle
%\makequote

%%%%%%%%%%%%%%%%%%%%%%%%%%%%%%%%%%%%%%%%%%%%%%%%%%%%%%%%%%%%%%%%%%%%%%%%%%%%%%%%%%%%%%%%%%%%%%%%%%%
\section{\textsc{Informação Pessoal}} 
\cvitem{Nome Completo}{João Pedro de Sousa Faria}
\cvitem{Data Nascimento}{30 de Maio de 1990}
\cvitem{Nacionalidade}{Portuguesa}
\cvitem{Naturalidade}{Santarém}
\cvitem{Morada}{Rua Câmara Pestana nº13  1ºDto, 1150-082, Lisboa}
\cvitem{Telefone}{966132388}
\cvitem{}{218884641}
\cvitem{Email}{\href{mailto:joaofaria90@gmail.com}{joaofaria90\MVAt gmail.com}}
%\cvitem{Email}{joaofaria90\MVAt gmail.com}
\cvitem{\textit{Webpage}}{\href{http://www.astro.up.pt/~jfaria}{http://www.astro.up.pt/\textasciitilde jfaria}}


%%%%%%%%%%%%%%%%%%%%%%%%%%%%%%%%%%%%%%%%%%%%%%%%%%%%%%%%%%%%


%%%%%%%%%%%%%%%%%%%%%%%%%%%%%%%%%%%%%%%%%%%%%%%%%%%%%%%%%%%%%%%%%%%%%%%%%%%%%%%%%%%%%%%%%%%%%%%%%%%%%%%%%%%%%%%%%
\section{\textsc{Formação Académica}}
\hspace{\fill}

\cventry
	{2011 - presente}
	{Mestrado em Astronomia (frequência 1º ano)}
	{\newline Departamento de Física e Astronomia}
	{Faculdade de Ciências da Universidade do Porto}
	{}
	{\hspace{\fill}}  % arguments 3 to 6 can be left empty

\vskip-1em

\cventry
	{2008 - 2011}
	{Licenciatura em Física – Ramo de Astronomia e Astrofísica} 
	{\newline Departamento de Física}
	{Faculdade de Ciências da Universidade de Lisboa}
	{\newline Média Final: $16$ em $20$ valores}
	{\hspace{\fill}}

\closesection{}


%%%%%%%%%%%%%%%%%%%%%%%%%%%%%%%%%%%%%%%%%%%%%%%%%%%%%%%%%%
%\section{\textsc{Unidades Curriculares Relevantes}}
%
%\vspace{0.2cm}
%
%	\cvitem{\textbf{Licenciatura:}}{}
%	\cvitem{Astronomia}{Escalas espaciais e temporais em Astronomia; astronomia esférica; magnitudes; produção de energia 					nas estrelas; classificação espectral; formação e evolução estelar; estrutura e dinâmica da Via Láctea; 					cefeidas; supernovas; AGN; quasares; cosmologia; estrutura e evolução do Universo}
%
%	\cvitem{Astrofísica}{Produção, emissão e absorção de radiação electromagnética; radiação térmica e radiação não-térmica; 					radiação de corpo negro; leis da radiação; espectros e atmosferas estelares; classes espectrais e 					classes de luminosidade; produção e transporte de energia; equações de estrutura estelar; formação e evolução estelar}

%	\cvitem{\textbf{Mestrado:}}{}
%	\cvitem{Galáxias}{Tipos e constituição; esquemas de classificação morfológica: Hubble, de Vaucouleurs, CAS; perfil de luminosidade; SEDs; o meio interestelar - detecção, propriedades e correlações; poeira e extinção; relações entre massa e luminosidade: Tully-Fisher, Faber-Jackson, plano fundamental; grupos e enxames de galáxias; segregação morfológica; efeito Butcher-Oemler; galáxias K+A; AGNs: taxonomia e propriedades observacionais; esquema de unificação}

\vspace{-0.5cm}
%%%%%%%%%%%%%%%%%%%%%%%%%%%%%%%%%%%%%%%%%%%%%%%%%%%%%%%%%%%%%%%%%
\section{\textsc{Prémios e Bolsas}}

	\cvitem{Bolsa de Investigação}{
	\textbf{ID Projecto:} PTDC/CTE-AST/098754/2008\newline 
	\textbf{T\'{i}tulo:} Probing Inside Binary Stars with Asteroseismology \newline
	\textbf{Supervisores:} Mário J. P. F. G. Monteiro; Margarida Cunha \newline
	\textbf{Instituição:} Centro de Astrofísica da Universidade do Porto (CAUP) \newline
	\textbf{Duração:} 1 Julho 2012 - 31 Dezembro 2012 (6 meses)
	}
	
	\cvlistitem{Olimpíadas da Astronomia - Edição 2007/2008 \newline
			\textbf{5º lugar na Final Nacional}}
	
	\vspace{0.2cm}
	
	\cvlistitem{Torneio Inter-Escolas CV-Robôs - Concurso de 2008 \newline
			\textbf{1º Classificado}}
	
\closesection{}

%%%%%%%%%%%%%%%%%%%%%%%%%%%%%%%%%%%%%%%%%%%%%%%%%%%%%%%%%%%%%%%%%%%%%%%%%%%%%%%%%%%%%%%%%%%%%%%%%%%%%%%%%%%%%%%%%%
%\section{\textsc{Experiência Profissional}}
%
%\cvitem{Cargo Ocupado}{\textbf{Accções de degustação / promoções em grandes superficies comerciais}} 
%\cvitem{Empregador}{Promoluz}
%\cvitem{Local Trabalho}{Lisboa}
%\cvitem{Sector}{Comercial, Vendas}
%
%\cvitem{Cargo Ocupado}{\textbf{Assistente de Call-Center}} 
%\cvitem{Empregador}{CCT}
%\cvitem{Local Trabalho}{Lisboa}
%\cvitem{Sector}{Call Center / HelpDesk}
%
%\cvitem{Outra Experiência}{\textbf{Empregado de balcão num bar de familia.}\newline
%	\textbf{Monitor de actividades de divulgação ciêntifica.}}
%\vspace{1cm}

%\vspace*{1cm}


%%%%%%%%%%%%%%%%%%%%%%%%%%%%%%%%%%%%%%%%%%%%%%%%%%%%%%%%%%%%%%%%%%%%%%%%%%%%%%%%%%%%%%%%%%%%%%%%%%%%%%%%%%%%%%%%%%%
%\section{\textsc{Experiência Profissional}}
%
%\cvitem{Cargo Ocupado}{\textbf{Ações de degustação / promoções em grandes superfícies comerciais}} 
%\cvitem{Empregador}{Promoluz}
%\cvitem{Local Trabalho}{Lisboa}
%\cvitem{Sector}{Comercial, Vendas}
%
%\cvitem{Cargo Ocupado}{\textbf{Assistente de Call-Center}} 
%\cvitem{Empregador}{CCT}
%\cvitem{Local Trabalho}{Lisboa}
%\cvitem{Sector}{Call Center / HelpDesk}
%
%\cvitem{Cargo Ocupado}{\textbf{Trabalho temporário em escritório e armazém}} 
%\cvitem{Empregador}{Randstad}
%\cvitem{Local Trabalho}{Lisboa}
%\cvitem{Sector}{Armazém | Escritório}
%
%\cvitem{Outra Experiência}{\textbf{Monitor de actividades de divulgação científica.}}
%%\vspace{1cm}

%\vspace*{-1cm}
%\section{\textsc{Carta de Condução}}
%
%	\cvitem{}{Possuo carta de condução de veículos ligeiros (categoria B)}
%\closesection{}



%%%%%%%%%%%%%%%%%%%%%%%%%%%%%%%%%%%%%%%%%%%%%%%%%%%%%%%%%%%%%%%%%%%%%%%%%%%%%%%%%%%%%%%%%%%%%%%%%%%%%%%%%%%%%%%%%
\section{\textsc{Seminários/Apresentações}}

\cvlistitem{\textit{\textsc{MESA} - Modules for Experiments in Stellar Astrophysics} 
\newline
João Faria, CAUP, DFA-UP \newline
16 January 2013, CAUP - Porto
}
\cvlistitem{\textit{Cores e Metalicidades de Galáxias Anãs Azuis no Enxame da Virgem} \newline
	João Faria, CAAUL, DF-FCUL \newline
	13 de Julho de 2011, Lisboa - CAAUL
}

\newpage
%%%%%%%%%%%%%%%%%%%%%%%%%%%%%%%%%%%%%%%%%%%%%%%%%%%%%%%%%%%%%%%%%%%%%%%%%%%%%%%%%%%%%%%%%%%%%%%%%%%%%%%%%%%%%%%%%%
\section{\textsc{Actividades Científicas}}
	
	\cvitem{Projecto de Licenciatura}{Durante o segundo semestre do 3º ano da Licenciatura, desenvolvi um projecto de investigação com o Doutor Marco Grossi no estudo de cores e metalicidades de galáxias anãs azuis no Enxame da Virgem. Neste trabalho obtive dados fotométricos e espectroscópicos (a partir do SDSS) para uma amostra de galáxias BCD. Com estes dados derivei valores para as metalicidades das galáxias, na tentativa de avaliar o efeito do ambiente denso do enxame na evolução destes objectos.}

	\cvitem{Posters}{}
	\vskip-2.5em
	\cvlistitem{\underline{Using seismic probes to study the core composition in pulsating white dwarfs}\newline
	- The Modern Era of Helio- and Asteroseismology (Obergurgl, Austria, May 2012)}
	

	\cvlistitem{\underline{Seismic diagnostics of the internal structure of 16 Cyg A and B}\newline
	- XXII National Astronomy and Astrophysics Meeting (Porto, September 2012)}
	
	
	\cvlistitem{\underline{Cores e metalicidades de galáxias anãs azuis no Enxame da Virgem}\newline
	- XXI National Astronomy and Astrophysics Meeting (Coimbra, September 2011)}


	
%%%%%%%%%%%%%%%%%%%%%%%%%%%%%%%%%%%%%%%%%%%%%%%%%%%%%%%%%%%%%%%%%%%%%%%%%%%%%%%%%%%%%%%%%%%%%%%%%%%%%%%%%%%%%%%%%%%
%\section{\textsc{Participação em Conferências e Workshops}}
%
%	\cvlistitem{XXI Encontro Nacional de Astronomia e Astrofísica\newline
%	7-10 Setembro 2011, Coimbra, Portugal} 
%		
%	\vspace{0.3cm}
%	
%	\cvlistitem{ALMA National Comunity Day \newline
%	27 de Abril de 2011, CAAUL - Lisboa} 
%		
%	\vspace{0.3cm}
%	
%	\cvlistitem{Joint European and National Astronomy Meeting (JENAM 2010) \newline
%	André Moitinho, Jan Palous \newline  
%	6 a 10 de Setembro de 2010, FCUL - Lisboa} 
%
%	\vspace{0.3cm}
%
%	\cvlistitem{Estágio "Astro-Heliofísica no Observatório Astronómico de Lisboa" \newline
%	Ciência nas Férias, Ciência Viva \newline  
%	3 e 4 de Setembro de 2007, OAL - Lisboa} 
%
%\closesection{}
%	

%%%%%%%%%%%%%%%%%%%%%%%%%%%%%%%%%%%%%%%%%%%%%%%%%%%%%%%%%%%%%%%%%%%%%%%%%%%%%%%%%%%%%%%%%%%%%%%%%%%%%%%%%%%%%%%%%%%
%\section{\textsc{Prémios}}
%	\cvlistitem{Olimpíadas da Astronomia - Edição 2007/2008 \newline
%			\textbf{5º lugar na Final Nacional}}
%	
%	\vspace{0.2cm}
%	
%	\cvlistitem{Torneio Inter-Escolas CV-Robôs - Concurso de 2008 \newline
%			\textbf{1º Classificado}}
%\closesection{}

%\newpage
%%%%%%%%%%%%%%%%%%%%%%%%%%%%%%%%%%%%%%%%%%%%%%%%%%%%%%%%%%%%%%%%%%%%%%%%%%%%%%%%%%%%%%%%%%%%%%%%%%%%%%%%%%%%%%%%%%%
%\section{\textsc{Interesses Científicos}}
%	\cvlistitem{Formação e Evolução de Galáxias}
%	\cvlistitem{Formação e Evolução Estelar}
%	\cvlistitem{Modelos Estelares}
%	\cvlistitem{Detecção de Exoplanetas}
%
%\closesection{}

%%%%%%%%%%%%%%%%%%%%%%%%%%%%%%%%%%%%%%%%%%%%%%%%%%%%%%%%%%%%%%%%%%%%%%%%%%%%%%%%%%%%%%%%%%%%%%%%%%%%%%%%%%%%%%%%%%%
\section{\textsc{Actividades Divulgação Científica}}

	\cvitem{Desde Janeiro de 2010}{Co-organizador da actividade pública mensal \textbf{Noites no Observatório} \newline
		 \textit{Observatório Astronómico de Lisboa}}
	
	\cventry{Desde Janeiro de 2010}{Co-organizador de actividades de divulgação científica}{\newline 
				      Observatório Astronómico de Lisboa}{}{}{
		\mybitem{Eclipse Total da Lua - 15 de Junho de 2011}
		\mybitem{Global Star Party - 09 de Abril de 2011}
		\mybitem{Observação da Lua - 19 de Março de 2011}
		\mybitem{A Lua Vermelha - 30 de Outubro de 2010}
	}
	
	\cvitem{Verão de 2010 e 2011}{Monitor do programa \textbf{Ciência Viva no Verão} \newline
		 \textit{Observatório Astronómico de Lisboa}}

	\cvitem{Outubro 2011}{Cordenador da actividade \textbf{Fim de semana nas estrelas}\newline
		 \textit{Associação Juvenil de Ciência (AJC) - Grupo de Imagem e Recriação Astronómica}}
		 
	\cvitem{Observação com telescópios}{Coordenei e participei em diversas acções de formação sobre montagem e utilização de telescópios no Observatório Astronómico de Lisboa, incluindo formação em reconhecimento do céu nocturno, alinhamento, astrofotografia com CCD, etc.}

%%%%%%%%%%%%%%%%%%%%%%%%%%%%%%%%%%%%%%%%%%%%%%%%%%%%%%%%%%%

%\section{\textsc{Referências}}
%
%\vspace{0.3cm}
%
%	\cvlistitem{\textbf{Prof. Doutor João Lin Yun} \newline
%			Email: \textit{yun@oal.ul.pt} \newline
%			Centro de Astronomia e Astrofísica da Universidade de Lisboa \newline
%			Observatório Astronómico de Lisboa \newline
%			Tapada da Ajuda - Edifício Leste - 2º Piso \newline
%			1349-018 Lisboa
%			}
%
%
%	\cvlistitem{\textbf{Doutor Marco Grossi} \newline
%			Email: \textit{grossi@oal.ul.pt} \newline
%			Centro de Astronomia e Astrofísica da Universidade de Lisboa \newline
%			Observatório Astronómico de Lisboa \newline
%			Tapada da Ajuda - Edifício Leste - 2º Piso \newline
%			1349-018 Lisboa
%}		
%\vspace{0.3cm}
%
%	\cvlistitem{\textbf{Doutora Catarina Lobo} \newline
%			Email: \textit{lobo@astro.up.pt} \newline
%			Centro de Astrofísica da Universidade do Porto \newline
%			Rua das Estrelas \newline
%			4150-762 Porto 
%			}

%%%%%%%%%%%%%%%%%%%%%%%%%%%%%%%%%%%%%%%%%%%%%%%%%%%%%%%%%%%%%%%%%%%%%%%%%%%%%%%%%%%%%%%%%%%%%%%%%%%%%%%%%%%%%%%%%%
\section{\textsc{Aptidões e Competências Linguísticas}}

	\cvitem{Português}{Língua Materna}
	\cvitem{Inglês}{Fluente (Expressão e Compreensão Escrita e Oral)}
	\cvitem{Francês}{Elementar}

\closesection{}

%\newpage
\vspace{-0.5cm}

%%%%%%%%%%%%%%%%%%%%%%%%%%%%%%%%%%%%%%%%%%%%%%%%%%%%%%%%%%%%%%%%%%%%%%%%%%%%%%%%%%%%%%%%%%%%%%%%%%%%%%%%%%%%%%%%%%%
\section{\textsc{Aptidões e Competências Informáticas}}

	\cvitem{Sist. Operativos}{Linux, Microsoft Windows\texttrademark}
	\cvitem{Programação}{C, \textsc{Matlab}, R  \newline Mathematica, IDL, Awk scripting (elementar)}
	\cvitem{Edição Texto}{\TeX  ( \LaTeX, BibTeX) , Microsoft Office, OpenOffice}
	%\cventry{Software}{}{}{}{}{
	\cvitem{Software}{
		\textbullet \hspace{0.01cm} Visualização/Edição Imagens: Photoshop \newline 
		\textbullet \hspace{0.01cm} Base de Dados: TopCat \newline 
		\textbullet \hspace{0.01cm} Produção de gráficos: Gnuplot, TopCat
		} 
%	\cvitem{Obtenção de Dados}{
%		SDSS (Sky Server/CasJobs),
%	$  $	Nasa Extragalactic Database,
%		ESO Archive}
		
\closesection{}



\end{document}


