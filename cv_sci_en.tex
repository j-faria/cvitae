%% start of file `jdoe_classic.tex'.
%% Copyright 2006 Xavier Danaux.
%
% This work may be distributed and/or modified under the
% conditions of the LaTeX Project Public License version 1.3c,
% available at http://www.latex-project.org/lppl/.


\documentclass[10pt]{moderncv}

% moderncv styles
% optional argument are 'blue' (default), 'orange', 'red', 'green', 'grey', 'nocolor' (for B&W) and 'roman' (for roman fonts, instead of sans serif fonts)
%\moderncvstyle{casual} 
\moderncvtheme[grey]{classic}

\usepackage[portuguese]{babel}	
% replace by the encoding you are using
\usepackage[utf8]{inputenc}

\usepackage{marvosym}		
% Package for the @ "at" symbol


% personal data (the given example is exhaustive; just give what you want)
\firstname{João}
\familyname{Faria}
\title{Curriculum Vit\ae{}}


%\photo[64pt]{../foto.jpg}  

%\renewcommand{\listsymbol}{{\fontencoding{U}\fontfamily{ding}\selectfont\tiny\symbol{'102}}} % define another symbol to be used in front of the list items

% the ConTeXt symbol
\def\ConTeXt{%
  C%
  \kern-.0333emo%
  \kern-.0333emn%
  \kern-.0667em\TeX%
  \kern-.0333emt}

% slanted small caps (only with roman family; the sans serif font doesn't exists :-()
%\usepackage{slantsc}
%\DeclareFontFamily{T1}{myfont}{}
%\DeclareFontShape{T1}{myfont}{m}{scsl}{ <-> cork-lmssqbo8}{}
%\usefont{T1}{myfont}{m}{scsl}Testing the font


% command and color used in this document, independently from moderncv 
\definecolor{links}{rgb}{0,0,1}% for web links
\hypersetup{colorlinks,linkcolor=,urlcolor=links}

\newcommand{\up}[1]{\ensuremath{^\textrm{\scriptsize#1}}}% for text subscripts

% teste para fazer lista dentro do cventry
\newcommand\mybitem[1]{%
   \parbox[t]{3mm}{\textbullet}\parbox[t]{10cm}{#1}\\[1.6mm]}

%----------------------------------------------------------------------------------
%            content
%----------------------------------------------------------------------------------
\begin{document}

\maketitle
%\makequote

%%%%%%%%%%%%%%%%%%%%%%%%%%%%%%%%%%%%%%%%%%%%%%%%%%%%%%%%%%%%%%%%%
\section{\textsc{Personal Information}}

\cvitem{Name}{João Pedro de Sousa Faria}
\cvitem{Born}{Santarém, Portugal, 30th of May 1990}
\cvitem{Nacionality}{Portuguese}
%\cvitem{Naturalidade}{Santarém}
\cvitem{Address}{Rua Câmara Pestana nº13  1ºDto, 1150-082, Lisboa}
\cvitem{Phone}{+351 966132388}
\cvitem{}{+351 218884641}
\cvitem{Email}{\href{mailto:joaofaria90@gmail.com}{joaofaria90\MVAt gmail.com}}
%\cvitem{Email}{joaofaria90\MVAt gmail.com}
\cvitem{\textit{Webpage}}{\href{http://www.astro.up.pt/\~jfaria}{http://www.astro.up.pt/\textasciitilde jfaria}}

%%%%%%%%%%%%%%%%%%%%%%%%%%%%%%%%%%%%%%%%%%%%%%%%%%%%%%%%%%%%%%%%%



%%%%%%%%%%%%%%%%%%%%%%%%%%%%%%%%%%%%%%%%%%%%%%%%%%%%%%%%%%%%%%%%%
\section{\textsc{Education}}

\cventry
	{2012 - present}
	{Master in Astronomy (dissertation)}
	{\newline Faculdade de Ciências \& Centro de Astrofísica (CAUP)}
	{Universidade do Porto}
	{\newline Project title: \emph{Probing inside binary stars with asteroseismology}
	\newline
	Supervisors: Mário J. P. F. G. Monteiro; Margarida Cunha}
	{}  % arguments 3 to 6 can be left empty


\cventry
	{2011 - 2012}
	{Master in Astronomy (curricular year)}
	{\newline Departamento de Física e Astronomia}
	{Faculdade de Ciências da Universidade do Porto}
	{Study plan can be acessed \href{https://sigarra.up.pt/fcup/pt/cur_geral.cur_planos_estudos_view?pv_plano_id=3421&pv_ano_lectivo=2012&pv_tipo_cur_sigla=&pv_origem=CUR}{here}; \newline 
	Average grade: $17$/$20$}
	{}  % arguments 3 to 6 can be left empty


\cventry
	{2008 - 2011}
	{\textit{Licenciatura} in Physics - Astronomy \& Astrophysics minor} 
	{\newline (Three year undergraduate course) \newline Departamento de Física}
	{Faculdade de Ciências da Universidade de Lisboa}
	{\newline Study plan can be acessed \href{http://fc.ul.pt/pt/cursos/licenciatura/fisica}{here}; \newline 
	Final average grade: $16$/$20$}
	{\hspace{\fill}}
%\cventry{2005 - 2008}
%	{Ensino Secundário - Curso de Ciências e Tecnologias}
%	{\newline Escola Secundária de Camões}
%	{Lisboa, Portugal}
%	{\newline Média Final: $17.3$ em $20$ valores}
%	{}
\closesection{}

%%%%%%%%%%%%%%%%%%%%%%%%%%%%%%%%%%%%%%%%%%%%%%%%%%%%%%%%%%%%%%%%%
\section{\textsc{Grants and Awards}}

	\cvitem{Research Scholarship}{
	\textbf{Project ID:} PTDC/CTE-AST/098754/2008\newline 
	\textbf{Title:} Probing Inside Binary Stars with Asteroseismology \newline
	\textbf{Supervisors:} Mário J. P. F. G. Monteiro; Margarida Cunha \newline
	\textbf{Institution:} Centro de Astrofísica da Universidade do Porto (CAUP) \newline
	\textbf{Duration:} 1 July 2012 - 31 December 2012 (6 months)
	}
	
\closesection{}

%%%%%%%%%
%\newpage 
%%%%%%%%%


%%%%%%%%%%%%%%%%%%%%%%%%%%%%%%%%%%%%%%%%%%%%%%%%%%%%%%%%%%%%%%%%%
\section{\textsc{Refereed Publications}}

	\cvitem{}{\emph{On the possibility of using seismic probes to study the core composition in pulsating white dwarfs}, Faria J.P., Monteiro M.J.P.F.G., Astron. Nachr. 333, 10, 954-957 (2012) \newline \textbf{DOI} 10.1002/asna.201211830}
%	\cvlistitem{"On the possibility of using seismic probes to study the core composition in pulsating white dwarfs", Faria J.P., Monteiro M.J.P.F.G., 2012, \emph{Astronomische Nachrichten} (in press)}
	
\closesection{}


%%%%%%%%%
\newpage 
%%%%%%%%%

%%%%%%%%%%%%%%%%%%%%%%%%%%%%%%%%%%%%%%%%%%%%%%%%%%%%%%%%%%%%%%%%%
%\section{\textsc{Astronomy-specific courses}}

%	\cvitem{Undergraduate:}
%	{The undergraduate degree in Physics provided me with a solid formation in many aspects of contemporary theoretical and experimental physics. Specific to astrophysics, the study plan}

%	\cvitem{Astrofísica}{Radiative theory; Stellar spectrometry and photometry; Theory of stellar structure and evolution; Theory of stellar formation.}

%	\cvitem{\textbf{Master:}}{}
%	\cvitem{Study Plan}{The Master in Astronomy at Faculdade de Ciências da Universidade do Porto provides a full academic first year, whose study plan can be acessed \href{https://sigarra.up.pt/fcup/pt/cur_geral.cur_planos_estudos_view?pv_plano_id=3421&pv_ano_lectivo=2012&pv_tipo_cur_sigla=&pv_origem=CUR}{here}.}


%%%%%%%%%%%%%%%%%%%%%%%%%%%%%%%%%%%%%%%%%%%%%%%%%%%%%%%%%%%%%%%%%
\section{\textsc{Scientific Activities}}
	
	\cvitem{Masters Dissertation}{Project in asteroseismology of the solar-type stars 16 Cyg A and B, using data from the \emph{Kepler} space telescope. By comparing different models of the structure of the stars using a larger set of diagnostic tools, beyond the common large and small separations, we aim to better isolate specific regions of the stellar interior and remove some of the effect that the ad-hoc surface corrections have on the modelling.
\newline
I am also studying the signal originated from acoustic glitches inside the stars and its detection both on second differences and the frequencies themselves.
%The use of diagnostic tools based on combinations of the oscillation frequencies to probe the inner core of the stars is being tested as a way to disentangle modelling degeneracies. 
}
	
	\cvitem{Undergraduate research project}{In the last semester of the undergraduate degree in Physics, I worked on a research project with Doctor Marco Grossi, studying the colours and metallicities of blue compact dwarf galaxies (BCD) in the Virgo Cluster. I analysed photometric and spectroscopic data from the SDSS for a sample of BCDs and derived estimates for the metallicities, trying to evaluate the effect of the dense cluster environment in the evolution of these objects.}

%Working with Doctor Marco Grossi, I presented two different posters on the colors and metallicities of BCD on the Física Ilimitada conference (FCUL, Lisbon, May 2011) and on the XXI National Astronomy and Astrophysics Meeting (Coimbra, September 2011). \newline 

	\cvitem{Conference Posters}{}

	\vspace{-2.5em}
	\cvlistitem{\underline{Cores e metalicidades de galáxias anãs azuis no Enxame da Virgem}\newline
	- XXI National Astronomy and Astrophysics Meeting (Coimbra, September 2011)
	\newline
	Supervised by Doctor Marco Grossi; final project of the undergraduate degree in Physics.}
	
	\vspace{0.3em}
	
	\cvlistitem{\underline{Using seismic probes to study the core composition in pulsating white dwarfs}\newline
	- The Modern Era of Helio- and Asteroseismology (Obergurgl, Austria, May 2012)
	\newline
	Together with Doctor Mário J. P. F. G. Monteiro; research project for a Masters course on stellar structure and evolution.}
	
	\vspace{0.3em}
	
	\cvlistitem{\underline{Seismic diagnostics of the internal structure of 16 Cyg A and B}\newline
	- XXII National Astronomy and Astrophysics Meeting (Porto, September 2012)
	\newline
	Supervised by Doctor Mário J. P. F. G. Monteiro and Doctor Margarida S. Cunha; Masters dissertation project.}



%%%%%%%%%%%%%%%%%%%%%%%%%%%%%%%%%%%%%%%%%%%%%%%%%%%%%%%%%%%%%%%%%
\section{\textsc{Seminars and Talks}}

\cvlistitem{\textit{Acoustic Glitches in 16 Cygni} 
\newline
6th Iberian Meeting on Asteroseismology \newline
27 May 2013, Aras de los Olmos - Spain
}

\cvlistitem{\textit{\textsc{MESA} - Modules for Experiments in Stellar Astrophysics} 
\newline
CAUP, DFA-UP \newline
16 January 2013, CAUP - Porto
}
\cvlistitem{\textit{Cores e Metalicidades de Galáxias Anãs Azuis no Enxame da Virgem} \newline
	CAAUL, DF-FCUL \newline
	13 July 2011, CAAUL - Lisbon
}


%%%%%%%%%%%%%%%%%%%%%%%%%%%%%%%%%%%%%%%%%%%%%%%%%%%%%%%%%%%%%%%%%
\section{\textsc{Scientific Interests}}
	\cvlistitem{Stellar Evolution; Stellar Convection and Modelling}
	%\cvlistitem{Stellar Convection Theory and Modelling}
	\cvlistitem{Computational Astrophysics}
	\cvlistitem{Star Formation}
	\cvlistitem{Exoplanets and host-stars}
	%\cvlistitem{Galactic Evolution}
\closesection{}
%%%%%%%%%%%%%%%%%%%%%%%%%%%%%%%%%%%%%%%%%%%%%%%%%%%%%%%%%%%%%%%%%%%%%%%%%%%%%%%%%%%%%%%%%%%%%%%%%%%%%%%%%%%%%%%%%%%

\section{\textsc{Conferences and Workshops}}

	\cvlistitem{6th Iberian Meeting on Asteroseismology\newline
	27-29 May 2013, Aras de los Olmos (Valencia), Spain} 
		
	\vspace{0.1cm}

	\cvlistitem{XXII Encontro Nacional de Astronomia e Astrofísica\newline
	23-25 September 2012, Porto, Portugal} 
		
	\vspace{0.1cm}

	\cvlistitem{The Modern Era of Helio- and Asteroseismology \newline 20-25 May 2012, Universitätszentrum Obergurgl, Austria }
	
	\vspace{0.1cm}
	
	\cvlistitem{XXI Encontro Nacional de Astronomia e Astrofísica\newline
	7-10 September 2011, Coimbra, Portugal} 
		
	\vspace{0.1cm}
	
	\cvlistitem{ALMA National Comunity Day \newline
	27 April 2011, CAAUL - Lisboa} 
		
	\vspace{0.1cm}
	
	\cvlistitem{Joint European and National Astronomy Meeting (JENAM 2010) \newline
	6-10 September 2010, FCUL - Lisboa} 

	\vspace{0.1cm}

%	\cvlistitem{Estágio "Astro-Heliofísica no Observatório Astronómico de Lisboa" \newline
%	Ciência nas Férias, Ciência Viva \newline  
%	3-4 September 2007, OAL - Lisboa} 

\closesection{}
	
%\newpage
%%%%%%%%%%%%%%%%%%%%%%%%%%%%%%%%%%%%%%%%%%%%%%%%%%%%%%%%%%%%%%%%%%%%%%%%%%%%%%%%%%%%%%%%%%%%%%%%%%%%%%%%%%%%%%%%%%%
%\section{\textsc{Other Awards}}
%	\cvlistitem{Olimpíadas da Astronomia - Edição 2007/2008 \newline
%			\textbf{5th place in the nacional finals}}
	
%	\vspace{0.2cm}
	
%	\cvlistitem{Torneio Inter-Escolas CV-Robôs - Concurso de 2008 \newline
%			\textbf{1st classified}}
%\closesection{}

%\newpage

%%%%%%%%%%%%%%%%%%%%%%%%%%%%%%%%%%%%%%%%%%%%%%%%%%%%%%%%%%%%%%%%%%%%%%%%%%%%%%%%%%%%%%%%%%%%%%%%%%%%%%%%%%%%%%%%%%%
\section{\textsc{Outreach}}

	\cvitem{From January 2010}{Co-organizer of monthly activity "\textbf{Noites no Observatório}" \newline
		 \textit{Observatório Astronómico de Lisboa}}
	
	\cventry{From January 2010}{Co-organizer of public outreach activities}{\newline 
				      Observatório Astronómico de Lisboa}{}{}{
		\mybitem{Eclipse Total da Lua - 15th July 2011}
		\mybitem{Global Star Party - 9th April 2011}
		\mybitem{Observação da Lua - 19th March 2011}
		\mybitem{A Lua Vermelha - 30th October 2010}
	}
	
	\cvitem{2010 and 2011 summer}{Monitor of the program "\textbf{Ciência Viva no Verão}"\newline
		 \textit{Observatório Astronómico de Lisboa}}
		 
	\cvitem{Amateur Observations}{I coordinated and participated in many workshops about observation and general use of amateur telescopes on the Astronomical Observatory of Lisbon. These include night sky recognition, telescope alignment  and CCD astrophotography.}

\closesection{}
	


%%%%%%%%%%%%%%%%%%%%%%%%%%%%%%%%%%%%%%%%%%%%%%%%%%%%%%%%%%%%%%%%%%%%%%%%%%%%%%%%%%%%%%%%%%%%%%%%%%%%%%%%%%%%%%%%%%
\section{\textsc{Language Skills}}

	\cvitem{Portuguese}{Native language}
	\cvitem{English}{Fluent - written and spoken}
	\cvitem{French}{Basic}

\closesection{}


%%%%%%%%%%%%%%%%%%%%%%%%%%%%%%%%%%%%%%%%%%%%%%%%%%%%%%%%%%%%%%%%%%%%%%%%%%%%%%%%%%%%%%%%%%%%%%%%%%%%%%%%%%%%%%%%%%%
\section{\textsc{Computer Skills}}

	\cvitem{Operating Systems}{Linux, Microsoft Windows}
	\cvitem{Programming}{Fortran, C, Python, \textsc{Matlab}, R, Awk (proficient) \newline Mathematica, IDL  (basic)}
	\cvitem{Office}{\TeX \hspace{0.01cm} (\LaTeX, BibTeX) , Microsoft Office, OpenOffice}
	%\cventry{Software}{}{}{}{}{
	\cvitem{Software}{
		\textbullet \hspace{0.01cm} Image Edition: Photoshop, GIMP\newline 
		\textbullet \hspace{0.01cm} Database: TopCat \newline 
		\textbullet \hspace{0.01cm} Plotting: Gnuplot, TopCat
		} 
		
	\cvitem{Specific}{Knowledge of the IRAF environment;
							 \newline
							 Basic knowledge of ESO's GASGANO pipeline;
							 \newline
							 Presented results with the MESA stellar evolution code.}
		
	\cvitem{Data Acquisition}{
		SDSS (Sky Server/CasJobs),
		Nasa Extragalactic Database,
		ESO Archive}

\closesection{}


%\section{\textsc{Referências}}
%
%\vspace{0.3cm}
%
%	\cvlistitem{\textbf{Prof. Doutor João Lin Yun} \newline
%			Email: \textit{yun@oal.ul.pt} \newline
%			Centro de Astronomia e Astrofísica da Universidade de Lisboa \newline
%			Observatório Astronómico de Lisboa \newline
%			Tapada da Ajuda - Edifício Leste - 2º Piso \newline
%			1349-018 Lisboa
%			}
%
%\vspace{0.3cm}
%
%	\cvlistitem{\textbf{Doutor Marco Grossi} \newline
%			Email: \textit{grossi@oal.ul.pt} \newline
%			Centro de Astronomia e Astrofísica da Universidade de Lisboa \newline
%			Observatório Astronómico de Lisboa \newline
%			Tapada da Ajuda - Edifício Leste - 2º Piso \newline
%			1349-018 Lisboa
%}		
%\vspace{10cm}
%
%	\cvlistitem{\textbf{Doutora Catarina Lobo} \newline
%			Email: \textit{lobo@astro.up.pt} \newline
%			Centro de Astrofísica da Universidade do Porto \newline
%			Rua das Estrelas \newline
%			4150-762 Porto 
%			}


\vfill
%%%%%%%%%%%%%%%%%%%%%%%%%%%%%%%%%%%%%%%%%%%%%%%%%%%%%%%%%%%%%%%%%%%%%%%%%%%%%%%%%%%%%%%%%%%%%%%%%%%%%%%%%%%%%%%%%%%
\section{\textsc{Updated}}
	\cvitem{}{June 6, 2013}


\end{document}


