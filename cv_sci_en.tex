%% start of file `jdoe_classic.tex'.
%% Copyright 2006 Xavier Danaux.
%
% This work may be distributed and/or modified under the
% conditions of the LaTeX Project Public License version 1.3c,
% available at http://www.latex-project.org/lppl/.


\documentclass[10pt]{moderncv}

% moderncv styles
% optional argument are 'blue' (default), 'orange', 'red', 'green', 'grey', 'nocolor' (for B&W) and 'roman' (for roman fonts, instead of sans serif fonts)
%\moderncvstyle{casual} 
\moderncvtheme[grey]{classic}

\usepackage[english]{babel}	
% replace by the encoding you are using
\usepackage[utf8]{inputenc}

\usepackage{marvosym}		
% Package for the @ "at" symbol
\usepackage{multicol} % For multiple column text.

% For \textscale, which I prefer over \sc (small caps).
% See the \acr command definition below.
\usepackage{relsize}

\usepackage{fancyhdr}
\pagestyle{fancy}
%\lfoot{center of the footer!}
\fancyfoot[l]{\addressfont\color{quotecolor}\footnotesize Curriculum Vitae -- João Faria}%


\usepackage{tgpagella}
\usepackage[expansion,protrusion]{microtype}
% Makes resume-specific commands available.
%\usepackage{resume}


% personal data (the given example is exhaustive; just give what you want)
\firstname{João}
\familyname{Faria}
\title{Curriculum Vit\ae{}}


%\photo[64pt]{../foto.jpg}  

% main title (name) with subtitle (date)
\newcommand\maintitle[3]{\vbox to 0pt{\hfill\scriptsize\color{gray} #3}\vspace{-0.4em}\noindent{\LARGE \textbf{#1}}\ \ \ \emph{#2}}

% \vspace variaties
\newcommand{\breakvspace}[1]{\pagebreak[2]\vspace{#1}\pagebreak[2]}
\newcommand{\nobreakvspace}[1]{\nopagebreak[4]\vspace{#1}\nopagebreak[4]}
% \spacedhrule a horizontal line with some vertical space before and after it
\newcommand{\spacedhrule}[2]{\breakvspace{#1}\hrule\nobreakvspace{#2}}

%\renewcommand{\listsymbol}{{\fontencoding{U}\fontfamily{ding}\selectfont\tiny\symbol{'102}}} % define another symbol to be used in front of the list items

% the ConTeXt symbol
\def\ConTeXt{%
  C%
  \kern-.0333emo%
  \kern-.0333emn%
  \kern-.0667em\TeX%
  \kern-.0333emt}

% slanted small caps (only with roman family; the sans serif font doesn't exists :-()
%\usepackage{slantsc}
%\DeclareFontFamily{T1}{myfont}{}
%\DeclareFontShape{T1}{myfont}{m}{scsl}{ <-> cork-lmssqbo8}{}
%\usefont{T1}{myfont}{m}{scsl}Testing the font

% pretty bullet (created from a much smaller centerdot), \sbull contains its spacing
\newcommand*\bull{\raisebox{-0.365em}[-1em][-1em]{\textscale{4}{$\cdot$}}}
\newcommand*\sbull{\ \ \bull \ \ }
% command and color used in this document, independently from moderncv 
\definecolor{links}{rgb}{0,0,1}% for web links
\hypersetup{colorlinks,linkcolor=,urlcolor=links}
% Setup the hyperrefs witht he right color.
\usepackage{hyperref}  
\usepackage{xcolor}
\definecolor{dark-blue}{rgb}{0.15,0.15,0.4}
\hypersetup{colorlinks,linkcolor={dark-blue},citecolor={dark-blue},urlcolor={dark-blue}}

% acr command, to quickly mark acronyms for special formatting
\newcommand*\acr[1]{\textscale{.85}{#1}}

\newcommand{\up}[1]{\ensuremath{^\textrm{\scriptsize#1}}}% for text subscripts

% teste para fazer lista dentro do cventry
\newcommand\mybitem[1]{%
   \parbox[t]{3mm}{\textbullet}\parbox[t]{10cm}{#1}\\[1.6mm]}

%----------------------------------------------------------------------------------
%            content
%----------------------------------------------------------------------------------
\begin{document}

%\maketitle
%\makequote

% title on top of the document
\maintitle{Jo\~{a}o Pedro de Sousa Faria}{PhD Student}{Last update on \today}

\nobreakvspace{0.3em}  % add some page break averse vertical spacing

% \noindent prevents paragraph's first lines from indenting
% \mbox is used to obfuscate the email address
% \sbull is a spaced bullet
% \href well..
% \\ breaks the line into a new paragraph
\noindent \href{mailto:joao.faria@astro.up.pt}{joao.faria\MVAt astro.up.pt}\sbull
\textsmaller{+}351 966132388\sbull
\href{http://www.astro.up.pt/\~jfaria}{www.astro.up.pt/\textasciitilde jfaria}
\\
30th of May 1990\sbull
Santarém, Portugal
\\
Largo Ferreira Lapa, 44, 3ºT, 4150-323, Porto



\spacedhrule{1.1em}{0.4em}  % a horizontal line with some vertical spacing before and after
%\roottitle{Summary}  % a root section title
\vspace{-1em}  % some vertical spacing
\begin{multicols}{2}  % open a multicolumn environment
%\noindent \emph{Creative geek with roots in the open source movement, an entrepreneurial mindset and a passion for delivering value by developing maintainable software.}
%\\
%\\
Holding an exemplary education record, 
with Bachelor and Master degree grades within the 25th and 10th top percentiles, 
I have acquired both theoretical and practical research experience in astrophysics.
While working on my MSc dissertation, 
I had the opportunity to strengthen my skills on data analysis and computational modeling methods,
as well as to present my work on international conferences and several talks, 
attend advanced lecture courses
and publish in peer-reviewed journals. 
More recently, I coordinated a team workshop on stellar evolution models, 
developed an open source code for the analysis of radial velocities 
and observed with ESO's HARPS instrument, in La Silla, Chile.

I am looking to secure a PhD position in the Center for Astrophysics of University of Porto, 
to continue my studies within the extrasolar planets team, to which I believe I can bring immediate and
strategic value.
\end{multicols}
\spacedhrule{0.4em}{0.4em}  % a horizontal line with some vertical spacing before and after

%%%%%%%%%%%%%%%%%%%%%%%%%%%%%%%%%%%%%%%%%%%%%%%%%%%%%%%%%%%%%%%%%
%\section{\textsc{Personal Information}}
%
%\cvitem{Name}{João Pedro de Sousa Faria}
%\cvitem{Born}{Santarém, Portugal, 30th of May 1990}
%\cvitem{Nacionality}{Portuguese}
%%\cvitem{Naturalidade}{Santarém}
%\cvitem{Address}{Rua Câmara Pestana nº13  1ºDto, 1150-082, Lisboa}
%\cvitem{Phone}{+351 966132388}
%\cvitem{}{+351 218884641}
%\cvitem{Email}{\href{mailto:joao.faria@astro.up.pt}{joao.faria\MVAt astro.up.pt}}
%%\cvitem{Email}{joaofaria90\MVAt gmail.com}
%\cvitem{Webpage}{\href{http://www.astro.up.pt/\~jfaria}{http://www.astro.up.pt/\textasciitilde jfaria}}

%%%%%%%%%%%%%%%%%%%%%%%%%%%%%%%%%%%%%%%%%%%%%%%%%%%%%%%%%%%%%%%%%



%%%%%%%%%%%%%%%%%%%%%%%%%%%%%%%%%%%%%%%%%%%%%%%%%%%%%%%%%%%%%%%%%
\section{\textsc{Education}}

\cventry
	{present}
	{Doctoral Program in Astronomy (doctoral course)}
	{\newline Faculdade de Ciências \& Centro de Astrofísica (CAUP)}
	{Universidade do Porto}
	{\newline 
	-- advanced courses on \emph{High Resolution Spectroscopy} and \emph{Adaptive Optics in Astronomy}}
	{}  % arguments 3 to 6 can be left empty


\cventry
	{2012 - 2013}
	{Master in Astronomy}
	{\newline Faculdade de Ciências \& Centro de Astrofísica (CAUP)}
	{Universidade do Porto}
	{\newline Project title: \emph{Asteroseismology of 16 Cyg A and B}
	\newline
	Supervisors: Mário J. P. F. G. Monteiro; Margarida Cunha}
	{Final average grade: $19$/$20$}  % arguments 3 to 6 can be left empty


%\cventry
%	{2011 - 2012}
%	{Master in Astronomy (curricular year)}
%	{\newline Departamento de Física e Astronomia}
%	{Faculdade de Ciências da Universidade do Porto}
%	{Study plan can be acessed \href{https://sigarra.up.pt/fcup/pt/cur_geral.cur_planos_estudos_view?pv_plano_id=3421&pv_ano_lectivo=2012&pv_tipo_cur_sigla=&pv_origem=CUR}{here}; \newline 
%	Average grade: $17$/$20$}
%	{}  % arguments 3 to 6 can be left empty


\cventry
	{2008 - 2011}
	{Bachelor in Physics - minor in Astronomy \& Astrophysics} 
	{\newline Departamento de Física}
	{Faculdade de Ciências da Universidade de Lisboa}
	{\newline Study plan can be acessed \href{http://www.fc.ul.pt/pt/cursos/licenciatura/fisica}{here}; \newline 
	Final average grade: $16$/$20$}
	{\hspace{\fill}}
%\cventry{2005 - 2008}
%	{Ensino Secundário - Curso de Ciências e Tecnologias}
%	{\newline Escola Secundária de Camões}
%	{Lisboa, Portugal}
%	{\newline Média Final: $17.3$ em $20$ valores}
%	{}
\closesection{}

%%%%%%%%%%%%%%%%%%%%%%%%%%%%%%%%%%%%%%%%%%%%%%%%%%%%%%%%%%%%%%%%%
\section{\textsc{Grants and Awards}}

	\cvitem{PhD Grant}{
	\textbf{Project ID:} CAUP-12/2013-BD (ERC-2009-StG-239953)\newline 
	\textbf{Title:} Exoplanet Detection in Metal-Poor Stars \newline
	\textbf{Supervisor:} Nuno C. Santos; Pedro Figueira\newline
	\textbf{Institution:} Centro de Astrofísica da Universidade do Porto (CAUP) \newline
	\textbf{Duration:} 1 October 2013 - 30 September 2014 (12 months)
	}
	

	\cvitem{S\&T Management Grant}{
	\textbf{Project ID:} CAUP-08/2013-BGCT\newline 
	\textbf{Supervisor:} Filipe Pires \newline
	\textbf{Institution:} Centro de Astrofísica da Universidade do Porto (CAUP) \newline
	\textbf{Duration:} 1 July 2013 - 30 September 2013 (3 months)
	}


	\cvitem{Research Scholarship}{
	\textbf{Project ID:} PTDC/CTE-AST/098754/2008\newline 
	\textbf{Title:} Probing Inside Binary Stars with Asteroseismology \newline
	\textbf{Supervisors:} Mário J. P. F. G. Monteiro; Margarida Cunha \newline
	\textbf{Institution:} Centro de Astrofísica da Universidade do Porto (CAUP) \newline
	\textbf{Duration:} 1 July 2012 - 31 December 2012 (6 months)
	}
	
\closesection{}

%%%%%%%%%
%\newpage 
%%%%%%%%%


%%%%%%%%%%%%%%%%%%%%%%%%%%%%%%%%%%%%%%%%%%%%%%%%%%%%%%%%%%%%%%%%%
%\section{\textsc{Submitted Publications}}
%
%
%\closesection{}


%%%%%%%%%
\newpage 
%%%%%%%%%

%%%%%%%%%%%%%%%%%%%%%%%%%%%%%%%%%%%%%%%%%%%%%%%%%%%%%%%%%%%%%%%%%
\section{\textsc{Refereed Publications}}

	\cvitem{}{\emph{The HARPS search for southern extra-solar planets: XXXV. 
	          The interesting case of HD41248: stellar activity, no planets?}, 
	          Santos, N. C., Mortier, A., \underline{Faria, J. P.}, et al., 2014 (accepted for publication in A\&A)}

	\cvitem{}{\emph{Asteroseismic estimate of helium abundance of a solar analog binary system}, Verma K., \underline{Faria, J. P.}, et al., 2014 (accepted for publication in ApJ)}

	\cvitem{}{\emph{On the possibility of using seismic probes to study the core composition in pulsating white dwarfs}, \underline{Faria, J. P.}, Monteiro M.J.P.F.G., Astron. Nachr. 333, 10, 954-957 (2012) \newline \textbf{\small DOI} 10.1002/asna.201211830}
%	\cvlistitem{"On the possibility of using seismic probes to study the core composition in pulsating white dwarfs", Faria J.P., Monteiro M.J.P.F.G., 2012, \emph{Astronomische Nachrichten} (in press)}
	
\closesection{}


%%%%%%%%%
%\newpage 
%%%%%%%%%

%%%%%%%%%%%%%%%%%%%%%%%%%%%%%%%%%%%%%%%%%%%%%%%%%%%%%%%%%%%%%%%%%
%\section{\textsc{Astronomy-specific courses}}

%	\cvitem{Undergraduate:}
%	{The undergraduate degree in Physics provided me with a solid formation in many aspects of contemporary theoretical and experimental physics. Specific to astrophysics, the study plan}

%	\cvitem{Astrofísica}{Radiative theory; Stellar spectrometry and photometry; Theory of stellar structure and evolution; Theory of stellar formation.}

%	\cvitem{\textbf{Master:}}{}
%	\cvitem{Study Plan}{The Master in Astronomy at Faculdade de Ciências da Universidade do Porto provides a full academic first year, whose study plan can be acessed \href{https://sigarra.up.pt/fcup/pt/cur_geral.cur_planos_estudos_view?pv_plano_id=3421&pv_ano_lectivo=2012&pv_tipo_cur_sigla=&pv_origem=CUR}{here}.}


%%%%%%%%%%%%%%%%%%%%%%%%%%%%%%%%%%%%%%%%%%%%%%%%%%%%%%%%%%%%%%%%%
\section{\textsc{Scientific Activities and Skills}}

	\cvitem{Observing Experience}{ESO 3.6m Telescope: HARPS spectrograph (January 2014)}

	\cvitem{MSc Dissertation}{Asteroseismology of the solar-type stars 16 Cyg A and B, using data from the \emph{Kepler} space telescope. Within this project I developed data analysis tools such as a software package to manipulate sets of oscillation frequencies and a genetic algorithm-based fitting routine. Also, I gained substantial expertise on the MESA stellar evolution code.
%	By comparing different models of the structure of the stars using seismic diagnostic tools, beyond the common large and small separations, we aimed to better isolate specific regions of the stellar interior and remove some of the effect that the ad-hoc surface corrections have on the modeling. 
%We also studied the signal originating from acoustic glitches inside the stars and its detection both on second differences and in the frequencies. 
%The use of diagnostic tools based on combinations of the oscillation frequencies to probe the inner core of the stars is being tested as a way to disentangle modelling degeneracies. 
}
	
	\cvitem{Undergraduate research project}{Research project with Dr. Marco Grossi, studying the colors and metallicities of blue compact dwarf galaxies (BCD) in the Virgo Cluster, using SDSS data. 
	%In the last semester of the undergraduate degree in Physics, I worked on a
	%I analysed photometric and spectroscopic data from the SDSS for a sample of BCDs and derived estimates for the metallicities, trying to evaluate the effect of the dense cluster environment in the evolution of these objects.
	}
%Working with Doctor Marco Grossi, I presented two different posters on the colors and metallicities of BCD on the Física Ilimitada conference (FCUL, Lisbon, May 2011) and on the XXI National Astronomy and Astrophysics Meeting (Coimbra, September 2011). \newline 

	\cvitem{Conference Posters}{}

	\vspace{-2.5em}
	
	\cvlistitem{\underline{Seismic diagnostics of the internal structure of 16 Cyg A and B}\newline
	- XXII National Astronomy and Astrophysics Meeting (Porto, September 2012)
	\newline
	Supervised by Doctor Mário J. P. F. G. Monteiro and Doctor Margarida S. Cunha; Masters dissertation project.}
	
	\vspace{0.3em}
	
	\cvlistitem{\underline{Using seismic probes to study the core composition in pulsating white dwarfs}\newline
	- The Modern Era of Helio- and Asteroseismology (Obergurgl, Austria, May 2012)
	\newline
	Together with Doctor Mário J. P. F. G. Monteiro; research project for a Masters course on stellar structure and evolution.}
	
	\vspace{0.3em}
	
	\cvlistitem{\underline{Cores e metalicidades de galáxias anãs azuis no Enxame da Virgem}\newline
	- XXI National Astronomy and Astrophysics Meeting (Coimbra, September 2011)
	\newline
	Supervised by Doctor Marco Grossi; final project of the undergraduate degree in Physics.}
	

	%%%%%%%%%%%%%
	\vspace{0.3em}
	
	\cvitem{Other skills}{Probability \sbull distributions; maximum likelihood estimation \newline 
	                      Frequentist statistics \sbull correlation tests; hypothesis testing; confidence intervals  \newline 
	                      Bayesian statistics\sbull sampling methods; MCMC; model comparison \newline
	                      Algorithms \sbull genetic and evolutionary algortihms; minimization; regression \newline
	                      Methods \sbull Monte Carlo, jacknife and bootstrap resampling; time series analysis}



%%%%%%%%%%%%%%%%%%%%%%%%%%%%%%%%%%%%%%%%%%%%%%%%%%%%%%%%%%%%%%%%%
\section{\textsc{Seminars and Talks}}

\cvlistitem{\textbf{\textit{The metal-poor survey}},
%\newline
CAUP Cookie Seminar\newline
23 April 2014, CAUP - Porto
}

\cvlistitem{\textbf{\textit{Asteroseismology of 16 Cyg A \& B}},
%\newline
MSc Thesis Defense\newline
30 October 2013, FCUP - Porto
}

\cvlistitem{\textbf{\textit{Acoustic Glitches in 16 Cygni}}, 
%\newline
XXIII National Astronomy and Astrophysics Meeting\newline
18 July 2013, FCUL - Lisbon
}

\cvlistitem{\textbf{\textit{Acoustic Glitches in 16 Cygni}}, 
%\newline
6th Iberian Meeting on Asteroseismology \newline
27 May 2013, Aras de los Olmos - Spain
}

\cvlistitem{\textbf{\textit{\textsc{MESA} - Modules for Experiments in Stellar Astrophysics}},
%\newline
CAUP Cookie Seminar \newline
16 January 2013, CAUP - Porto
}
\cvlistitem{\textbf{\textit{Cores e Metalicidades de Galáxias Anãs Azuis no Enxame da Virgem}}, 
%\newline
CAAUL, DF-FCUL \newline
13 July 2011, CAAUL - Lisbon
}


%%%%%%%%%
%\newpage 
%%%%%%%%%

%%%%%%%%%%%%%%%%%%%%%%%%%%%%%%%%%%%%%%%%%%%%%%%%%%%%%%%%%%%%%%%%%
\section{\textsc{Broad Scientific Interests}}
	\cvlistitem{Exoplanets and host-stars}
	%\cvlistitem{Stellar Convection Theory and Modelling}
	\cvlistitem{Computational Astrophysics}
	\cvlistitem{Stellar Evolution; Stellar Convection and Modelling}
	\cvlistitem{Star Formation}

	%\cvlistitem{Galactic Evolution}
\closesection{}
%%%%%%%%%%%%%%%%%%%%%%%%%%%%%%%%%%%%%%%%%%%%%%%%%%%%%%%%%%%%%%%%%%%%%%%%%%%%%%%%%%%%%%%%%%%%%%%%%%%%%%%%%%%%%%%%%%%

\section{\textsc{Conferences and Workshops}}

	\cvlistitem{XXIII National Astronomy and Astrophysics Meeting (oral presentation)\newline
	18-19 July 2013, Lisbon, Portugal}
		
	\vspace{0.1cm}


	\cvlistitem{6th Iberian Meeting on Asteroseismology (oral presentation)\newline
	27-29 May 2013, Aras de los Olmos (Valencia), Spain} 
		
	\vspace{0.1cm}

	\cvlistitem{XXII National Astronomy and Astrophysics Meeting (poster presentation)\newline
	23-25 September 2012, Porto, Portugal} 
		
	\vspace{0.1cm}

	\cvlistitem{The Modern Era of Helio- and Asteroseismology (poster presentation) \newline 20-25 May 2012, Universitätszentrum Obergurgl, Austria }
	
	\vspace{0.1cm}
	
	\cvlistitem{XXI National Astronomy and Astrophysics Meeting\newline
	7-10 September 2011, Coimbra, Portugal} 
		
	\vspace{0.1cm}
	
	\cvlistitem{ALMA National Comunity Day \newline
	27 April 2011, CAAUL - Lisbon} 
		
	\vspace{0.1cm}
	
	\cvlistitem{Joint European and National Astronomy Meeting (JENAM 2010) \newline
	6-10 September 2010, FCUL - Lisbon} 

	\vspace{0.1cm}

%	\cvlistitem{Estágio "Astro-Heliofísica no Observatório Astronómico de Lisboa" \newline
%	Ciência nas Férias, Ciência Viva \newline  
%	3-4 September 2007, OAL - Lisboa} 

\closesection{}
	
%\newpage
%%%%%%%%%%%%%%%%%%%%%%%%%%%%%%%%%%%%%%%%%%%%%%%%%%%%%%%%%%%%%%%%%%%%%%%%%%%%%%%%%%%%%%%%%%%%%%%%%%%%%%%%%%%%%%%%%%%
%\section{\textsc{Other Awards}}
%	\cvlistitem{Olimpíadas da Astronomia - Edição 2007/2008 \newline
%			\textbf{5th place in the nacional finals}}
	
%	\vspace{0.2cm}
	
%	\cvlistitem{Torneio Inter-Escolas CV-Robôs - Concurso de 2008 \newline
%			\textbf{1st classified}}
%\closesection{}

%\newpage

%%%%%%%%%%%%%%%%%%%%%%%%%%%%%%%%%%%%%%%%%%%%%%%%%%%%%%%%%%%%%%%%%%%%%%%%%%%%%%%%%%%%%%%%%%%%%%%%%%%%%%%%%%%%%%%%%%%
\section{\textsc{Outreach}}

	\cvitem{September 2013}{Monitor of the summer program "\textbf{Escola de Ver\~{a}o de F\'{i}sica}" \newline 
	     \textit{Faculdade de Ciências da Universidade do Porto}}
	
	\cvitem{Summer 2013}{Collaborator of Outreach Unit (planetarium presenter) \newline \textit{Centro de Astrofisica da Universidade do Porto}}

	\cvitem{2010 - 2011}{Co-organizer of monthly activity "\textbf{Noites no Observatório}" \newline
		 \textit{Observatório Astronómico de Lisboa}}
	
	\cvitem{2010 - 2011}{Co-organizer of public outreach activities\newline 
	\textit{Observatório Astronómico de Lisboa}
%		\begin{itemize}
%		\item Eclipse Total da Lua - 15th July 2011
%		\item Global Star Party - 9th April 2011
%		\end{itemize}
		%\mybitem{Global Star Party - 9th April 2011}
		%\mybitem{Observação da Lua - 19th March 2011}
		%\mybitem{A Lua Vermelha - 30th October 2010}
	}
	
	\cvitem{Summer 2010/2011}{Monitor of the program "\textbf{Ciência Viva no Verão}"\newline
		 \textit{Observatório Astronómico de Lisboa}}
		 
	\cvitem{Amateur Observations}{Coordinated and participated in many workshops about observation and general use of amateur telescopes on the Astronomical Observatory of Lisbon. These include night sky recognition, telescope alignment  and CCD astrophotography. Also participated in other night sky observation activities.}

\closesection{}
	

%%%%%%%%%
\newpage 
%%%%%%%%%

%%%%%%%%%%%%%%%%%%%%%%%%%%%%%%%%%%%%%%%%%%%%%%%%%%%%%%%%%%%%%%%%%%%%%%%%%%%%%%%%%%%%%%%%%%%%%%%%%%%%%%%%%%%%%%%%%%
\section{\textsc{Language Skills}}

	\cvitem{Portuguese}{Native speaker}
	\cvitem{English}{Proficient user (C1)}
	\cvitem{French}{Basic}
	\cvitem{Spanish}{Basic}

\closesection{}


%%%%%%%%%%%%%%%%%%%%%%%%%%%%%%%%%%%%%%%%%%%%%%%%%%%%%%%%%%%%%%%%%%%%%%%%%%%%%%%%%%%%%%%%%%%%%%%%%%%%%%%%%%%%%%%%%%%
\section{\textsc{Computer Skills}}

	\cvitem{OS}{Linux, Microsoft Windows}
	\cvitem{Programming}{Python, Fortran (advanced) \newline
	                     C, \textsc{Matlab}, R, Awk (proficient) \newline Mathematica, IDL  (basic)}
	\cvitem{Office}{\TeX \hspace{0.01cm} (\LaTeX, BibTeX), Microsoft Office, OpenOffice}
	%\cventry{Software}{}{}{}{}{
	\cvitem{Software}{
		\textbullet \hspace{0.01cm} Image Edition: Photoshop, GIMP\newline 
		\textbullet \hspace{0.01cm} Database/VO: TopCat \newline 
		\textbullet \hspace{0.01cm} Plotting: Gnuplot, TopCat
		} 
		
	\cvitem{Specific}{Knowledge of the IRAF environment, PyRAF;
					  \newline
					  Knowledge of ESO's GASGANO pipeline;
					  \newline
					  Advanced knowledge of the MESA stellar evolution code.}
		
	\cvitem{Data Acquisition}{
		SDSS (Sky Server/CasJobs),
		Nasa Extragalactic Database,
		ESO Archive}

\closesection{}


%\section{\textsc{Referências}}
%
%\vspace{0.3cm}
%
%	\cvlistitem{\textbf{Prof. Doutor João Lin Yun} \newline
%			Email: \textit{yun@oal.ul.pt} \newline
%			Centro de Astronomia e Astrofísica da Universidade de Lisboa \newline
%			Observatório Astronómico de Lisboa \newline
%			Tapada da Ajuda - Edifício Leste - 2º Piso \newline
%			1349-018 Lisboa
%			}
%
%\vspace{0.3cm}
%
%	\cvlistitem{\textbf{Doutor Marco Grossi} \newline
%			Email: \textit{grossi@oal.ul.pt} \newline
%			Centro de Astronomia e Astrofísica da Universidade de Lisboa \newline
%			Observatório Astronómico de Lisboa \newline
%			Tapada da Ajuda - Edifício Leste - 2º Piso \newline
%			1349-018 Lisboa
%}		
%\vspace{10cm}
%
%	\cvlistitem{\textbf{Doutora Catarina Lobo} \newline
%			Email: \textit{lobo@astro.up.pt} \newline
%			Centro de Astrofísica da Universidade do Porto \newline
%			Rua das Estrelas \newline
%			4150-762 Porto 
%			}


\vfill
%%%%%%%%%%%%%%%%%%%%%%%%%%%%%%%%%%%%%%%%%%%%%%%%%%%%%%%%%%%%%%%%%%%%%%%%%%%%%%%%%%%%%%%%%%%%%%%%%%%%%%%%%%%%%%%%%%%%
%\section{\textsc{Updated}}
%\selectlanguage{english}
%	\cvitem{}{\today}


\end{document}


